\documentclass[utf8]{beamer}
\usetheme{Boadilla}

\title{Lexical Text Simplification}
\author{Paula Gombar, Ivan Katanić}
\institute{FER}
\date{\today}

\begin{document}

\begin{frame}
\titlepage
\end{frame}

\begin{frame}
\frametitle{Contents}
\tableofcontents
\end{frame}

\section{Task definition}

\begin{frame}
\frametitle{SemEval-2012 Task 1}
\begin{itemize}
 \item finding less complex semantically-appropriate words or phrases and replacing those that are difficult to comprehend
  \item common pipeline in such a system:
  \begin{enumerate}
    \item \textbf{complexity analysis:} finding out which words or phrases are considered complex
    \item \textbf{substitute lookup:} retrieving adequate replacements, simpler than the original word or phrase
    \item \textbf{context-based ranking:} ranking of substitutes to produce the final replacement
  \end{enumerate}
\end{itemize}
\end{frame}

\begin{frame}
\frametitle{Example}
Sentence: \textit{The incisions will feel constricted for the first 24-48 hours.} \\
We identify the word \textit{constricted} as complex, retrieve the possible substitutes: \\
\{uncomfortable\} \{tight\} \{stretched\} \{compressed\} \{constricted\}. \\
We score each candidate on simplicity and context-adequacy, rank them and determine the simplest one, e.g. \textit{tight}.
\end{frame}

\section{Features extracted}

\begin{frame}
\frametitle{Features extracted}
\begin{itemize}
\item \textbf{inverse word length}
\item \textbf{number of synsets in WordNet}
\begin{itemize}
 \item i.e. the word \textit{fundamental} has the following synsets:
 \begin{enumerate}
  \item (n) fundamental (any factor that could be considered important to the understanding of a particular business)
  \item (n) fundamental, fundamental frequency, first harmonic (the lowest tone of a harmonic series)
 \end{enumerate}
\end{itemize}
\item \textbf{frequency in Simple Wikipedia}
\item \textbf{frequency in Wikipedia}
\item \textbf{corpus complexity}
\[ C_w = \frac{f_{w, English}}{f_{w, Simple}} \]
where $f_{w, c}$ is the frequency of candidate $w$ in corpus $c$. 
\end{itemize}
\end{frame}

\begin{frame}
\begin{itemize}
\frametitle{Features extracted}
\item \textbf{context similarity}
\[ csim(w, c) = \sum_{w' \in C(w)} cos(\mathbf{v_c}, \mathbf{v_{w'}}) \]
where $C(w)$ is the set of context words of the original word $w$ and $\mathbf{v_c}$ is the GloVe vector of the replacement candidate $c$. \\
\item \textbf{semantic similarity}
\[ ssim(w, c) = cos(\mathbf{v_w}, \mathbf{v_c}) \]
where $\mathbf{v_w}$ is the GloVe vector of the original word $w$.
\end{itemize}
\end{frame}

\section{Methods used}

\begin{frame}
\frametitle{Methods used}
\textbf{Ranking SVM with RBF kernel.}

\begin{table}
\caption{Optimal hyperparameters for Ranking SVM with RBF kernel.}
\label{tab:rbf-params}
\begin{center}
\begin{tabular}{l | l | l}
Hyperparameter & Optimal value & Possible values \\
\hline \hline
Scaler & Standard &  Standard, MinMax, None \\
PolyFeatures degree & 1 & 1, 2  \\
$C$ & $2^5$ & $[2^{-15}, ..., 2^{8}]$ \\
$\gamma$ & 0.00098 & $[2^{-15}, ..., 2^{8}]$ \\
\end{tabular}
\end{center}
\end{table}

\textbf{Ranking SVM with linear kernel.}

\begin{table}
\caption{Optimal hyperparameters for Ranking SVM with linear kernel.}
\label{tab:rbf-params}
\begin{center}
\begin{tabular}{l | l | l}
Hyperparameter & Optimal value & Possible values \\
\hline \hline
Scaler & Standard &  Standard, MinMax, None \\
PolyFeatures degree & 1 & 1, 2  \\
$C$ & $2^4$ & $[2^{-15}, ..., 2^{8}]$
\end{tabular}
\end{center}
\end{table}
\end{frame}

\begin{frame}
\frametitle{Methods used}
\textbf{Linear combination of features.}

\begin{table}
\caption{Optimal hyperparameters for linear combination of features.}
\label{tab:linear-params}
\begin{center}
\begin{tabular}{l | l}
Feature & Weight \\
\hline \hline
Inverse word length & 1 \\
WordNet synsets & 0 \\
Simple Wikipedia frequency & 10 \\
English Wikipedia frequency & 0 \\
Corpus complexity & 0 \\
Context similarity & 9 \\
Semantic similarity & 0 \\
\end{tabular}
\end{center}
\end{table}

\textbf{Unsupervised approach.} Scale the data using MinMax scaler, declare all coefficients as 1.
\end{frame}

\section{Evaluation}
\begin{frame}
 \frametitle{Baselines}
\textbf{L-Sub Gold.} This baseline uses the gold-standard annotations from the Lexical Substitution corpus of SemEval-2007 as is. \\
\textbf{Random} Randomizes the order, allowing ties. \\
\textbf{Simple Freq.} Uses the frequency of the substitutes as extracted from the Google Web 1T Corpus.

\begin{table}
\caption{Baseline kappa scores on Trial and Test datasets.}
\label{tab:baseline}
\begin{center}
\begin{tabular}{l|l|l}
Baseline & Trial & Test \\
\hline \hline
L-Sub Gold & 0.050 & 0.106 \\
Random & 0.016 & 0.012 \\
Simple Freq. & \textbf{0.397} & \textbf{0.471} \\
\end{tabular}
\end{center}
\end{table}

\end{frame}

\begin{frame}
 \frametitle{Results}

\begin{table}
\caption{Implemented methods kappa scores on the Test dataset.}
\label{tab:res}
\begin{center}
\begin{tabular}{l|l}
Method name & Test score \\
\hline \hline
Ranking SVM with RBF kernel & \textbf{0.461} \\
Ranking SVM with linear kernel & 0.443 \\
Linear combination of features & 0.459 \\
Unsupervised approach & 0.313 \\
\end{tabular}
\end{center}
\end{table}
\end{frame}

\begin{frame}
\frametitle{Conclusion}
\begin{itemize}
 \item four different methods, using both context-dependent and context-independent features, as well as external resources such as state-of-the-art word vector representations and simplified corpora.
  \item the performance of supervised approaches is likely to improve with larger training sets
  \item very strong relation between distributional frequency of words and their perceived simplicity
\end{itemize}
\end{frame}

\begin{frame}
 \frametitle{The end}
\begin{center}
 Thank you! Any questions?
\end{center}
\end{frame}

\end{document}
